\documentclass[11pt]{article}
\usepackage[utf8]{inputenc}
\usepackage[T1]{fontenc}
\usepackage{acl2012}
\usepackage{times}
\usepackage{latexsym}
\usepackage{amsmath}
\usepackage{multirow}
\usepackage{url}
\usepackage[english]{babel}
\usepackage{enumerate}
\usepackage{graphicx}
\usepackage{fancyhdr}
\usepackage{booktabs}
\usepackage{float}
\usepackage{hyperref}
\usepackage{listings}
\usepackage{listings}
\usepackage[norelsize,ruled,vlined]{algorithm2e}


\DeclareMathOperator*{\argmax}{arg\,max}
\setlength\titlebox{6.5cm}    % Expanding the titlebox
\frenchspacing


% Custom commands
\newcommand{\class}[1]{\sloppy{\texttt{#1}}}

\newenvironment{definitions}[1]%
{\begin{list}{}{\settowidth{\labelwidth}{\textbf{#1}}
            \setlength{\leftmargin}{\labelwidth}
            \addtolength{\leftmargin}{\labelsep}
    \renewcommand{\makelabel}[1]{\textbf{\hfill##1}}}}%
{\end{list}}

\title{
    Fun and functional? An experiment in crowdsourcing named entity tags\\
    (UROP 12 ECTS Project Proposal)\thanks{This would constitue a final project (ísl. Lokaverkefni) for all three members according to above order: BSc in Discrete Mathematics and Computer Science, Diploma in Applied Computing and BSc in Computer Science.}
}

\author{Ólafur Páll Geirsson, Guðmundur Harðarsson, Guðmundur Þ Guðmundsson \\
    School of Computer Science \\ 
    Reykjavik University \\
    % Menntavegur 1, 101 Reykjavik \\
    {\tt \{olafurpg11,gudmundurhar12,gudmundurthg\}@ru.is}
  }

\date{}



%%%%%%%%%%%%%%%%%%%%%%%%%%%%%%%%%%%%%%%%%%%%%%%%%%%%%%%%%%%%%
%                        Setup
%%%%%%%%%%%%%%%%%%%%%%%%%%%%%%%%%%%%%%%%%%%%%%%%%%%%%%%%%%%%%


\begin{document}
\maketitle


\begin{abstract}
    Many natural language processing tasks rely on having large labeled data
    sets at hand.  For less commonly spoken languages, such may not necessarily
    exists and tools for collecting them appear in limited numbers. In this
    project, we propose to use the concept of a “game with a purpose” for
    collecting large amounts of named entity tags for Icelandic language text.
\end{abstract}

\section{Introduction}
The size of available annotated data sets in a given language has an large
impact on the performance of many core natural language tasks (e.g.\@
\cite{banko_scaling_2001}). While large-scale annotation tasks for natural
language processing research on commonly spoken languages (e.g. English,
Chinese, Arabic) is becoming a feasible option (e.g.\@ \cite{snow_cheap_2008})
with the introduction of commercial crowdsourcing services (e.g. Amazon
Mechanical Turk, CrowdFlower), less commonly spoken languages (e.g. Icelandic)
are left out with no similar option. 

The ESP game \cite{von_ahn_labeling_2004} introduced a novel approach to
labeling large data sets. In the game, users are paired up and asked to label
an image depending on its content. If the users agree on the input, they score
points. The incentive for users to take part in the game is no more than the
pleasure of playing it, as opposed to the financial reward in commercial
crowdsourcing services. Through the lifetime of the ESP game, a total of
1,271,451 labels were generated for 293,760 images which then were released for
use to further improve image recognition algorithms. This success inspired a new genre of
games commonly referred to as “games with a purpose” (GWAP), popularized by Luis von Ahn
\shortcite{von_ahn_human_2009}. Various attempts (e.g.
\cite{von_ahn_improving_2006,von_ahn_peekaboom:_2006-1,scharl_leveraging_2012,siorpaes_games_2008})
have been made to repeat the idea on other kinds of labeling tasks with varying
results.

In this proposed UROP project, we aim to make an attempt to use the GWAP
concept for labeling named entities (e.g. Iceland as a location, CCP as an organization) in Icelandic text. To our best knowledge,
this has not been done before. Ahn and Dabbish \shortcite{von_ahn_designing_2008} offer a guide
to designing a GWAP. They go into three game “templates” that have shown to be
particularly successful, i.e.\@, output agreement games, inversion-problem
games, and input-agreement games. We intend to adopt one of these game
templates while meeting the following requirements:
\begin{itemize}
    \item Feature rich interface for the researcher, including the ability to
        \begin{itemize}
            \item work with large amounts of text, minimum 250.000 words.
            \item be language independent
            \item upload data sets of various formats
            \item customize taxonomy
            \item configure the data reliability, e.g. a confidence score for
                annotation correctness, this includes defining how to measure
                the data reliability.
            \item see statistics on annotation progress
            \item export annotated data
        \end{itemize}
    \item User friendly interface for the annotators, this includes
        developing
        \begin{itemize}
            \item an interesting and visually attractive gameplay
            \item a motivation system (e.g. scoring, awards, recognition)
        \end{itemize}
    \item Browser based, cross-platform, both desktop and hand-held devices
\end{itemize}

In developing this game, we set out to answer the following questions
\begin{itemize}
    \item How many human work hours can be mobilized with the game?
    \item How do these hours compare with the amount of hours put into
        developing the game itself?
    \item How efficient is the game with relation the number of accurately
        labeled named entities per human hour?
    \item Which scheme is most successful in obtaining reliable named entity
        labels? (e.g.\@  How many times does a named entity have to be labeled
        for it to become reliably accurate?)
    \item Are certain kinds of texts more enjoyable to label than others?
\end{itemize}

\section{Methodology}
The project will be split into three phases:
\begin{enumerate}
 \item Design the game, inspired by related work in GWAP and conventions in
     named entity recognition research.
 \item Develop the game using modern HTML5 web technologies.
 \item Perform an experimental study to evaluate the game. In this we would use
     the MIM-GOLD corpus \cite{helgadottir_tagged_2012}. Furthermore, we will
     use the labelled data to train and evaluate a named entity recognizer.
     Optionally, this named entity recognizer can be incorporated into a web
     service for demonstration purposes.
\end{enumerate}

\section{Goal}
By doing this project, we aim to meet these following goals
\begin{itemize}
    \item Release the source code of the game, which may turn out to be useful
        in collecting related annotations on other text corpora.
    \item Release named entity tags for a minimum of 250.000 words from the
        MIM-GOLD corpus.
    \item Release a named entity recognizer for Icelandic language text trained
        on the labeled data
    \item If successful, it may be possible to present the game at the Language
        Resources and Evaluation Conference (LREC) in May, 2014, held in Harpan
        Conference Centre, Reykjavik, e.g. during coffee breaks or in the
        corridors.
\end{itemize}


\section{Benefits}
With the project, we also consider the following benefits
\begin{itemize}
    \item Create a pathway for future Icelandic information extraction systems,
        e.g. in healthcare and financial sectors.
    \item Increase awareness of Icelandic language technology.
    \item A new tool for use in language technology research to collect
        annotations on large amounts of text data through gamification and
        crowdsourcing.
\end{itemize}
\section{Learning outcomes}
We seek to improve our knowledge and skills within various areas of computer science, these include:
\begin{itemize}
    \item Designing and developing a server and client side for a web-based game which
        is both enjoying to play while being useful for a researcher. This, we
        believe will be one of the key success factors of the project.
    \item Applying scientific methods to collect reliable data for use in
        research.
    \item Extend our knowledge and competence in developing data-driven natural
        language technologies.
    \item Direct and manage our own independent learning.
\end{itemize}

\bibliographystyle{acl2012}
\bibliography{proposal}
\end{document}



